%% homework_template.tex
%%
%% This template is a verbose starting point for mathematics, physics, and
%% engineering homework submissions. It includes:
%%   - A reusable header block
%%   - Problem and solution formatting
%%   - Proof environments
%%   - Enumerated sub-parts
%%   - Aligned equations

%--------------------------------------------------------------------------------------------------
% 1) Document class and packages
%--------------------------------------------------------------------------------------------------
\documentclass[12pt]{article}

% Page layout and core math packages.
\usepackage[margin=1in]{geometry}
\usepackage{amsmath,amsthm,amssymb}
\usepackage{enumitem}
\usepackage{mathtools}

%--------------------------------------------------------------------------------------------------
% 2) Theorem-like environments and custom helpers
%--------------------------------------------------------------------------------------------------
\newtheorem*{remark}{Remark}

% Keep all assignment metadata in one place for easy reuse.
\newcommand{\hwname}{John Doe}
\newcommand{\hwcourse}{Math 101}
\newcommand{\hwnum}{1}
\newcommand{\hwdate}{\today}

% Optional helper to visually separate problems.
\newcommand{\problemsep}{\vspace{0.6cm}\hrule\vspace{0.6cm}}

%--------------------------------------------------------------------------------------------------
% 3) Document body
%--------------------------------------------------------------------------------------------------
\begin{document}

% Header block (course, assignment number, student, date)
\begin{center}
  {\Large\textbf{\hwcourse: Homework \hwnum}} \\
  {\large\hwname} \\
  {\large\hwdate}
\end{center}

% Add any submission notes here, such as collaboration policy.
\noindent\textbf{Submission notes:} Show complete work, justify each step,
and cite any external references used.

\problemsep

%--------------------------------------------------------------------------------------------------
% Problem 1: Proof-based question
%--------------------------------------------------------------------------------------------------
\section*{Problem 1}
\textbf{Prompt.} Prove that $\sqrt{2}$ is irrational.

\textbf{Solution.}
\begin{proof}
Suppose, for contradiction, that $\sqrt{2}$ is rational. Then there exist
integers $a,b$ with $b\neq 0$ and $\gcd(a,b)=1$ such that
\[
\sqrt{2} = \frac{a}{b}.
\]
Squaring both sides gives
\[
2 = \frac{a^2}{b^2}
\quad\Longrightarrow\quad
a^2 = 2b^2.
\]
Hence $a^2$ is even, so $a$ is even. Write $a=2k$ for some integer $k$. Then
\[
(2k)^2 = 2b^2
\quad\Longrightarrow\quad
4k^2 = 2b^2
\quad\Longrightarrow\quad
b^2 = 2k^2.
\]
Thus $b^2$ is even, so $b$ is even. This implies both $a$ and $b$ are even,
contradicting $\gcd(a,b)=1$. Therefore $\sqrt{2}$ is irrational.
\end{proof}

\begin{remark}
This structure (assume rational, derive contradiction via parity) is a common
proof strategy for irrationality of square roots of nonsquare integers.
\end{remark}

\problemsep

%--------------------------------------------------------------------------------------------------
% Problem 2: Computation with aligned equations
%--------------------------------------------------------------------------------------------------
\section*{Problem 2}
\textbf{Prompt.} Compute the derivative of $f(x)=x^2\sin(x)$.

\textbf{Solution.}
Use the product rule $\frac{d}{dx}[uv]=u'v+uv'$ with
$u(x)=x^2$ and $v(x)=\sin(x)$:
\begin{align*}
f'(x)
  &= \frac{d}{dx}[x^2]\,\sin(x) + x^2\,\frac{d}{dx}[\sin(x)] \\
  &= 2x\sin(x) + x^2\cos(x).
\end{align*}

\problemsep

%--------------------------------------------------------------------------------------------------
% Problem 3: Multi-part example
%--------------------------------------------------------------------------------------------------
\section*{Problem 3}
\textbf{Prompt.} Let $g(x)=e^{2x}$. Answer the following:
\begin{enumerate}[label=(\alph*)]
  \item Compute $g'(x)$.
  \item Find the equation of the tangent line at $x=0$.
\end{enumerate}

\textbf{Solution.}
\begin{enumerate}[label=(\alph*)]
  \item By the chain rule,
  \[
  g'(x)=2e^{2x}.
  \]

  \item First compute the point and slope at $x=0$:
  \[
  g(0)=e^0=1,
  \qquad
  g'(0)=2e^0=2.
  \]
  Using point-slope form,
  \[
  y-1=2(x-0)
  \quad\Longrightarrow\quad
  y=2x+1.
  \]
\end{enumerate}

%--------------------------------------------------------------------------------------------------
% End of document
%--------------------------------------------------------------------------------------------------
\end{document}
