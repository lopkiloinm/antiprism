%% beamer_template.tex
%%
%% This template demonstrates a complete Beamer slide deck with explanatory
%% comments for each major block. It is intended for class presentations,
%% thesis defenses, project updates, and conference talks.
%%
%% Tip: If you are new to Beamer, keep this file as a reference and remove
%% sections gradually as your own slides take shape.

%--------------------------------------------------------------------------------------------------
% 1) Document class and presentation mode
%--------------------------------------------------------------------------------------------------
% Beamer supports options such as aspectratio=169 for widescreen slides.
\documentclass[aspectratio=169]{beamer}

%--------------------------------------------------------------------------------------------------
% 2) Theme and visual customization
%--------------------------------------------------------------------------------------------------
% Common themes: Madrid, CambridgeUS, Frankfurt, Boadilla, AnnArbor.
\usetheme{Madrid}
\usecolortheme{default}

% Optional package imports for math, graphics, and tables.
\usepackage{amsmath,amssymb}
\usepackage{graphicx}
\usepackage{booktabs}

% Metadata shown on title page and often in slide footers.
\title{Presentation Title}
\subtitle{A Verbose Beamer Starter Template}
\author{Author Name}
\institute{Institution or Laboratory Name}
\date{\today}

%--------------------------------------------------------------------------------------------------
% 3) Outline frame at the beginning of every section
%--------------------------------------------------------------------------------------------------
% This hook automatically inserts a "Section Overview" slide whenever a new
% \section starts.
\AtBeginSection[]
{
  \begin{frame}{Section Overview}
    \tableofcontents[currentsection]
  \end{frame}
}

%--------------------------------------------------------------------------------------------------
% 4) Document body
%--------------------------------------------------------------------------------------------------
\begin{document}

% Title slide
\begin{frame}
  \titlepage
\end{frame}

% Agenda slide
\begin{frame}{Agenda}
  \tableofcontents
\end{frame}

\section{Introduction}

\begin{frame}{Motivation}
  \begin{itemize}
    \item Explain the real-world problem your presentation addresses.
    \item Summarize why this topic matters to your audience.
    \item State what decisions, insights, or actions should come out of this talk.
  \end{itemize}
\end{frame}

\begin{frame}{Problem Statement}
  A clear problem statement keeps your talk focused:
  \begin{block}{Example Problem Statement}
    Existing workflows are slow, error-prone, and difficult to reproduce.
    We need a method that improves accuracy while reducing manual effort.
  \end{block}
\end{frame}

\section{Method}

\begin{frame}{Approach Overview}
  \begin{enumerate}
    \item Define requirements and success metrics.
    \item Design and implement the proposed solution.
    \item Evaluate outcomes against baseline methods.
  \end{enumerate}
\end{frame}

\begin{frame}{Mathematical Formulation}
  Beamer supports full \LaTeX{} math mode:
  \begin{equation*}
    J(\theta) = \frac{1}{n} \sum_{i=1}^{n} \ell\big(f_\theta(x_i), y_i\big)
  \end{equation*}
  where $J(\theta)$ is the objective, $\ell$ is a loss function, and
  $f_\theta$ is a model parameterized by $\theta$.
\end{frame}

\begin{frame}{Including Figures}
  Use \texttt{\textbackslash includegraphics} for diagrams or results:
  \begin{figure}
    \centering
    % Replace with your own figure file path.
    \fbox{\rule{0pt}{1.4in}\rule{0.8\linewidth}{0pt}}
    \caption{Placeholder for architecture diagram or experiment setup.}
  \end{figure}
\end{frame}

\section{Results}

\begin{frame}{Quantitative Results}
  \begin{table}
    \centering
    \begin{tabular}{lcc}
      \toprule
      Method & Accuracy (\%) & Runtime (s) \\
      \midrule
      Baseline & 82.4 & 4.8 \\
      Proposed & 89.7 & 3.1 \\
      \bottomrule
    \end{tabular}
    \caption{Sample benchmark comparison table.}
  \end{table}
\end{frame}

\begin{frame}{Key Takeaways}
  \begin{itemize}
    \item The proposed method improves quality metrics over baseline methods.
    \item Runtime is reduced, making the approach practical for daily use.
    \item Remaining limitations define future work and open research questions.
  \end{itemize}
\end{frame}

\section{Conclusion}

\begin{frame}{Conclusion and Future Work}
  \begin{block}{Conclusion}
    This template demonstrates how to structure a complete Beamer presentation
    with title, agenda, section overviews, methods, results, and final summary.
  \end{block}

  \begin{alertblock}{Future Work}
    Replace placeholders with domain-specific content and tailor visuals,
    pacing, and level of detail to your audience.
  \end{alertblock}
\end{frame}

\begin{frame}{Questions}
  \centering
  {\Large Thank you!} \\
  \vspace{0.4cm}
  Questions and discussion welcome.
\end{frame}

\end{document}
