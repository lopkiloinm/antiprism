%% notes_template.tex
%%
%% This template provides a verbose lecture-notes structure for courses in
%% mathematics, computer science, and engineering. It demonstrates:
%%   - Chapter and section organization
%%   - Definitions, lemmas, and theorems
%%   - Highlighted note boxes
%%   - Equations and proof-style writing

%--------------------------------------------------------------------------------------------------
% 1) Document class and packages
%--------------------------------------------------------------------------------------------------
\documentclass[11pt]{report}

% Geometry controls margins. One-inch margins are a practical default.
\usepackage[margin=1in]{geometry}

% Core math packages.
\usepackage{amsmath,amsthm,amssymb}

% Colored boxes for emphasis notes and reminders.
\usepackage{tcolorbox}
\usepackage{xcolor}

% Hyperlinks in PDF output (table of contents entries, references, URLs).
\usepackage[colorlinks=true,linkcolor=blue,citecolor=blue,urlcolor=blue]{hyperref}

%--------------------------------------------------------------------------------------------------
% 2) Theorem environments and note box definition
%--------------------------------------------------------------------------------------------------
\newtheorem{theorem}{Theorem}[chapter]
\newtheorem{lemma}[theorem]{Lemma}
\newtheorem{proposition}[theorem]{Proposition}
\newtheorem{definition}[theorem]{Definition}

% Reusable highlighted note box.
\newtcolorbox{note}[1][]{
  colback=gray!5,
  colframe=gray!55,
  fonttitle=\bfseries,
  title=#1
}

%--------------------------------------------------------------------------------------------------
% 3) Metadata
%--------------------------------------------------------------------------------------------------
\title{Lecture Notes}
\author{Your Name}
\date{Fall 2024}

%--------------------------------------------------------------------------------------------------
% 4) Document body
%--------------------------------------------------------------------------------------------------
\begin{document}

\maketitle
\tableofcontents
\newpage

\chapter{Introduction}
\section{How to Use These Notes}
These notes are designed to be both a reference and a learning guide. Each
chapter introduces ideas, states formal definitions, and develops intuition
with examples and short proofs.

\begin{note}[Study Tip]
When reviewing, focus first on definitions and theorem statements, then return
to proofs and examples to internalize why each result is true.
\end{note}

\section{Foundational Example}
\begin{definition}[Vector Space]
A vector space over a field $F$ is a set $V$ equipped with vector addition and
scalar multiplication satisfying closure, associativity, commutativity of
addition, additive identity and inverses, and compatibility with scalar
multiplication.
\end{definition}

\begin{proposition}
If $u,v\in V$ for a vector space $V$, then $u+v\in V$.
\end{proposition}

\begin{proof}
This is immediate from the closure axiom of vector addition in the definition
of a vector space.
\end{proof}

\chapter{Core Results}
\section{A Classical Geometric Theorem}
\begin{theorem}[Pythagorean Theorem]
In a right triangle with legs $a,b$ and hypotenuse $c$,
\[
a^2+b^2=c^2.
\]
\end{theorem}

\begin{note}[Interpretation]
The theorem relates lengths in Euclidean geometry and appears in algebra,
trigonometry, and numerical methods.
\end{note}

\section{Algebraic Lemma Example}
\begin{lemma}
For any real numbers $x,y$, we have
\[
(x+y)^2 = x^2 + 2xy + y^2.
\]
\end{lemma}

\begin{proof}
Expand directly:
\[
(x+y)^2=(x+y)(x+y)=x^2+xy+yx+y^2=x^2+2xy+y^2.
\]
\end{proof}

\chapter{Worked Practice}
\section{Short Guided Example}
Suppose $f(x)=x^3-3x$. Then
\[
f'(x)=3x^2-3=3(x^2-1).
\]
Critical points occur when $f'(x)=0$, i.e., at $x=\pm1$. This style of short,
step-by-step calculation is useful for revision notes.

\begin{note}[Template Extension]
You can split these notes into multiple files using
\texttt{\textbackslash include\{chapter1\}} and
\texttt{\textbackslash include\{chapter2\}} when the document grows.
\end{note}

\chapter*{Appendix: Common Commands}
\addcontentsline{toc}{chapter}{Appendix: Common Commands}
\begin{itemize}
  \item Inline math: \verb|$a^2+b^2=c^2$|
  \item Display math: \verb|\[ ... \]| or \verb|\begin{equation} ... \end{equation}|
  \item Theorems: \verb|\begin{theorem} ... \end{theorem}|
  \item References: \verb|\label{key}| and \verb|\ref{key}|
\end{itemize}

\end{document}
