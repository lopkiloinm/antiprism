%% book_template.tex
%%
%% This template provides a verbose starting point for writing books in LaTeX.
%% It demonstrates front matter, main matter, back matter, chapter structure,
%% equations, tables, figures, indexing, and cross-referencing.

%--------------------------------------------------------------------------------------------------
% 1) Document class and page layout
%--------------------------------------------------------------------------------------------------
\documentclass[11pt]{book}

% Geometry controls page margins. Typical book-like margins are often wider
% on the inside edge for binding; this simple template uses symmetric margins.
\usepackage[margin=1in]{geometry}

%--------------------------------------------------------------------------------------------------
% 2) Core packages
%--------------------------------------------------------------------------------------------------
\usepackage{amsmath,amssymb}
\usepackage{graphicx}
\usepackage{booktabs}
\usepackage{hyperref}
\usepackage{makeidx}

% Enables index generation. Compile with:
% pdflatex -> makeindex -> pdflatex -> pdflatex
\makeindex

%--------------------------------------------------------------------------------------------------
% 3) Metadata
%--------------------------------------------------------------------------------------------------
\title{The Title of the Book}
\author{Author Name}
\date{\today}

%--------------------------------------------------------------------------------------------------
% 4) Document body
%--------------------------------------------------------------------------------------------------
\begin{document}

% FRONT MATTER uses roman page numbers and is typically unnumbered in chapters.
\frontmatter
\maketitle

\chapter*{Dedication}
This optional unnumbered page is often used for a dedication.

\chapter*{Preface}
The preface introduces the motivation and audience for the book. It can also
explain how chapters are organized and suggest reading paths for different
types of readers.

\tableofcontents

% MAIN MATTER switches to arabic page numbers and numbered chapters.
\mainmatter

\chapter{Foundations}
\section{Why this topic matters}
This chapter introduces core ideas and notation used throughout the book. Use
clear definitions early so readers can follow later arguments efficiently.

\subsection{A first mathematical example}
Displayed equations are ideal for central formulas:
\begin{equation}
S_n = \sum_{k=1}^{n} k = \frac{n(n+1)}{2}
\label{eq:sum-formula}
\end{equation}
Equation~\ref{eq:sum-formula} will be automatically numbered and can be
referenced anywhere in the text.

\section{Including figures}
Figures help communicate complex structures quickly.
\begin{figure}[htbp]
  \centering
  % Replace this placeholder with your own file, e.g.:
  % \includegraphics[width=0.7\textwidth]{chapter1-diagram.pdf}
  \fbox{\rule{0pt}{1.6in}\rule{0.75\textwidth}{0pt}}
  \caption{Placeholder figure for a chapter diagram or conceptual map.}
  \label{fig:chapter-figure}
\end{figure}

\section{Including tables}
Tables are useful for concise comparisons.
\begin{table}[htbp]
  \centering
  \caption{Example comparison table used in a book chapter.}
  \label{tab:comparison}
  \begin{tabular}{lcc}
    \toprule
    Method & Accuracy (\%) & Time (s) \\
    \midrule
    Baseline & 81.2 & 5.4 \\
    Improved & 88.9 & 3.7 \\
    \bottomrule
  \end{tabular}
\end{table}

\chapter{Applications}
\section{Case Study}
This chapter demonstrates how foundational concepts apply in practice. Use
subsections for datasets, methodology, results, and discussion.

\subsection{Cross-references and indexing}
Cross-referencing improves readability in long documents. For example, see
Figure~\ref{fig:chapter-figure} and Table~\ref{tab:comparison}.

Important terms can be indexed for quick lookup, such as
\emph{cross-referencing}\index{cross-referencing},
\emph{book structure}\index{book structure}, and
\emph{chapter organization}\index{chapter organization}.

\chapter{Advanced Topics}
\section{Advice for scaling your manuscript}
As your manuscript grows, keep one chapter per file and include them with
\verb|\include{chapter-name}| or \verb|\input{chapter-name}| from a central
\texttt{main.tex}. This keeps compilation and collaboration manageable.

% BACK MATTER typically contains appendices, bibliography, and index.
\backmatter

\chapter{Appendix A: Supplementary Material}
Use appendices for proofs, implementation details, or large tables that would
interrupt the flow of the main chapters.

\chapter*{Bibliography Notes}
For production documents, use BibTeX or biblatex instead of manual references.
This template omits a full bibliography block to keep the structure focused.

% Prints the generated index.
\printindex

\end{document}
