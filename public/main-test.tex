% Automated edit timestamp: 2026-02-20T00:48:07.124Z

\documentclass[11pt]{article}
\usepackage[margin=1in]{geometry}
\usepackage{amsmath}
\usepackage{graphicx}
\usepackage{tikz-cd}
\usepackage{multicol}
\usepackage{booktabs}

\setlength{\parindent}{0pt}
\setlength{\parskip}{1\baselineskip}

\begin{document}

\section*{What is Prism? (Updated)}

\textbf{Prism} is an AI-powered \LaTeX{} editor for writing scientific documents. It supports real-time collaboration with coauthors and includes OpenAI-powered intelligence to help you draft and edit text, reason through ideas, and handle formatting.

\section*{What is Antiprism?}

\textbf{Antiprism} is a P2P decentralized version of Prism. Where Prism relies on cloud services and centralized infrastructure, Antiprism runs entirely in your browser---no servers, no API keys, no data leaving your device. Everything that makes Antiprism ``anti'' to Prism is summarized below.

\section*{Antiprism vs.\ Prism: Architecture Comparison}

\begin{center}
\begin{tabular}{@{}lll@{}}
\toprule
\textbf{Component} & \textbf{Prism (cloud)} & \textbf{Antiprism (client-side)} \\
\midrule
Realtime collaboration & WebSockets via central server & WebRTC + Yjs (peer-to-peer) \\
AI assistant & OpenAI API (datacenter) & LFM2.5-1.2B Q4 ONNX (WebGPU) \\
\LaTeX{} rendering & Cloud compilation & Client-side WASM (texlyre-busytex) \\
Data storage & Server-side & IndexedDB, local-first \\
\bottomrule
\end{tabular}
\end{center}

\subsection*{Realtime: WebRTC Yjs vs.\ WebSockets}

Prism uses WebSockets to sync edits through a central server. Antiprism uses \textbf{WebRTC} with \textbf{Yjs} for CRDT-based collaboration: peers connect directly to each other, and a signaling server (or public mesh) only helps establish connections---it never sees document content. Your edits sync peer-to-peer without a middleman.

\subsection*{AI: WebGPU in-Browser vs.\ OpenAI Datacenter}

Prism sends your text to OpenAI's servers. Antiprism runs \textbf{LFM2.5-1.2B Q4 ONNX} (LiquidAI) entirely in your browser via \textbf{WebGPU}. The model is quantized to 4-bit (Q4) and exported to ONNX for efficient in-browser inference. No API keys, no network calls for inference---the model loads once, caches locally, and runs on your GPU. Privacy and offline use come by design.

\subsection*{LaTeX: Client-Side WASM vs.\ Cloud}

Prism compiles \LaTeX{} in the cloud. Antiprism uses \textbf{texlyre-busytex}, a WebAssembly port of a \TeX{} engine, to compile and render PDFs locally. Your documents never leave your machine for compilation.

\section*{Features}

\begin{multicols}{2}
Antiprism includes an in-browser AI assistant and can access your project, so you can ask it to do things like:

``Add the equation for the Laplace transform of $t\cos(at)$ to the introduction.''
\[
  \mathcal{L}\left\{ t \cos(a t) \right\} = \frac{ s^2 - a^2 }{ (s^2 + a^2)^2 }
\]

``Add a 4-by-4 table'' to the summary section.
\begin{center}
\resizebox{0.5\linewidth}{!}{%
\begin{tabular}{|c|c|c|c|}
  \hline
  1 & 2 & 3 & 4 \\
  \hline
  5 & 6 & 7 & 8 \\
  \hline
  9 & 10 & 11 & 12 \\
  \hline
  13 & 14 & 15 & 16 \\
  \hline
\end{tabular}%
}
\end{center}

``Proofread this and highlight any errors or gaps in logic, and make suggestions for how I can improve the clarity of the section.''

``Are there any corollaries or follow-on implications of Theorem 3.1 that I've missed? Are all the bounds tight, or can some be relaxed?''

\columnbreak

``Write an abstract based on the rest of the paper''

``Add a bibliography to my paper, and suggest related work I may have missed.''

``Generate this hand-drawn diagram in \LaTeX{}.''
\par\noindent
\begin{minipage}[t]{0.49\linewidth}
  \vspace{0pt}
  \centering
  \includegraphics[width=\linewidth]{diagram.jpg}
\end{minipage}\hfill
\begin{minipage}[t]{0.49\linewidth}
  \vspace{0pt}
  \centering
  \resizebox{\linewidth}{!}{$
    \begin{tikzcd}[row sep=2em, column sep=1.5em, ampersand replacement=\&]
      E
        \arrow[dr, "e"']
        \arrow[drr, "p_2"]
        \arrow[ddr, "p_1"']
      \& \& \\
      \& A \times B \arrow[r, "\pi_2"'] \arrow[d, "\pi_1"] \& B \arrow[d, "g"] \\
      \& A \arrow[r, "f"'] \& C
    \end{tikzcd}
  $}
\end{minipage}
\par

``Add any missing dependencies across my project.''

``Generate a 200-word summary for a popular audience.''

\end{multicols}

\subsection*{Automated Testing}

This subsection was automatically added to test the diff calculation system. It demonstrates how the system handles structural changes to LaTeX documents.

\begin{itemize}
\item Automated diff generation
\item Forward and reverse engineering
\item Real-time highlighting
\end{itemize}

\section*{Collaboration}

Invite collaborators by clicking the ``Share'' menu. As you edit, they will see your updates in real time over WebRTC---no central server stores your document. You can also leave comments by highlighting text and selecting ``Leave a comment.''

\end{document}
